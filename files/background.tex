% chktex-file 46

\section{Background}
The formulation of the polarization problem we study focuses specifically on mathematical models of opinion formation in social networks. 

Social networks arise randomly, with individuals developing connections based on social groups. Thus, the natural mathematical representation of a social network is a graph. Each node represents an individual and an edge represents a friendship or connection in the network. Edges have weights associated with the extent of their friends' influence on their opinions-- in other words, the ``closeness" of the friendship. In a real-life social network like Facebook, a higher edge weight between users corresponds to a increased interactions between those users (e.g. their respective stories pop up in their news feeds more often). 

\subsection{The Stochastic Block Model}

In this project, we use a random graph model termed the Stochastic Block Mode (SBM) to construct a graph representing a social network. The SBM is a common generative model seen in numerous fields, including statistics, theoretical computer science, and machine learning \cite{sbm}. We refer the interested reader to \cite{sbm} for a comprehensive survey of applications of the SBM. 

An SBM graph $G$ has $n$ vertices, divided into several communities.
The probability that there is an edge between two nodes in the same community
is $p$ while the probability there is an edge between two nodes
in different communities is $q$.

Often, we have that $p \geq q$, to model the idea that relationships
within communities are more likely than relationships between
communities.
However, there are some real-world situations where $p \leq q$.
Consider a patient-doctor network and the way opinions of
e.g. a vaccine are formed.
Connections representing conversations about a vaccine
are most likely between a doctor and a patient and less
likely between patient and patient or doctor and doctor.
Such a model is particularly relevant to
the acceptance of COVID-19 vaccines \cite{hoff2020patient-doctor}. When $p=q$, the SBM is known as an Erdős-Rényi graph.

\subsection{Opinion Formation}
There are two main models of opinion formation in social
networks.
The input to both models is a social network
given by a graph and an innate (starting) opinion for each
node in the graph.
That is, each node $i$ has an innate opinion
$s_i$ and expressed opinion (that changes according
to the respective model) $z_i$
where $s_i, z_i \in [-1, 1]$.

\paragraph{DeGroot's Model:} Perhaps the most popular model of opinion dynamics studied, DeGroot's model updates a node's opinion with the weighted average of its expressed opinion and its friends opinions. Formally,
\begin{align}
    z_i^{(t)} = \frac{z_i^{(t-1)} + \sum_{j \neq i} w_{ij}z_j^{(t-1)}}{d_i + 1}
    \nonumber
\end{align}

where $z_i^{(t)}$ is the opinion of node $i$ at iteration $t$,
$z_i^{(t-1)}$ is the opinion of node $i$ in the previous iteration,
and $d_i$ is the degree of node $i$ in the graph.
\medskip \medskip

\paragraph{Friedkin-Johnsen Model:} The Friedkin-Johnsen (FJ) opinion dynamics model is similar to DeGroot's Model but, instead
of averaging with the node's own expressed opinion, averages
with the innate opinion.
Intuitively, FJ opinions are more `stubborn.'
Formally,

\begin{align}\label{eq:updateruleFJ}
    z_i^{(t)} = \frac{s_i + \sum_{j \neq i}
    w_{ij}z_j^{(t-1)}}{d_i + 1}
\end{align}

where $z_i^{(t)}$ is the opinion at iteration $t$, $s_i$ is the innate opinion of node $i$, and $d_i$ is the degree of node $i$ in the graph.

We will use FJ dynamics because
\cite{Dandekarpnas} have shown that polarization
(formally defined in the next section)
always converges in DeGroot's model.

\subsection{Measuring Polarization}

One measure of polarization with a natural mathematical interpretation is is the \emph{network disagreement} index. In \cite{Dandekarpnas}, a process is viewed as \emph{polarizing} if it increases the network disagreement index. In a social network graph $G = {V,E,w}$, the network disagreement index is calculated as follows:

\begin{align}
    \eta(G,x) = \sum_{i,j \in E} w_{ij}(x_i - x_j)^2 
    \nonumber
\end{align}

where $x_i$ and $x_j$ are the opinions at nodes $i$ and $j$.

In \cite{chitra20analyzing}, Chitra and Musco add some structure to the measure of variance. They define polarization as the \emph{variance} of a set of opinions. For a vector of $n$ opinions ${z} \in [-1,1]$, let

\begin{align}
    \E[{z}] = \frac{1}{n}\sum_{j=1}^n x_j 
    \nonumber
\end{align}

where $z_j$ is the mean opinion of {z}. Then, polarization, $P_z$ can be defined as:

\begin{align}
    P_z = \sum_{i=1}^n (x_i - \E[{z}])^2
    \nonumber
\end{align}

Another measure is \emph{bimodality}.
Instead of simply of measuring the difference from the mean,
bimodality gives a sense for whether opinions are centered
at two different means.
Formally, the bimodality coefficient is given by

\begin{align}
    \beta = \frac{\gamma^2 + 1}{\kappa}
    \nonumber
\end{align}

\noindent where $\gamma$ is the \emph{skewness} (third standardized moment) and $\kappa$ is the \emph{kurtosis} (fourth standardized moment). 
Measuring bimodality is particularly relevant to
the modern political situation in the United States
\cite{bromley2017tale}.

\subsection{Linear Algebraic Interpretation of Polarization} 
To analyze polarization on an SBM graph, it is helpful to have a linear algebraic interpretation of polarization. We follow the approach in  \cite{chitra20analyzing}.

Let $A \in R^{n \times n}$ be the adjacency matrix of $G$, where $A_{ij} = A_{ji}$ since the graph is undirected. Let $D$ be a diagonal matrix where $D_{ii} = d_i$. That is, each diagonal entry is equal to the degree of node $i$, and the other entries are 0's. Finally, let $L = D - A$ be the \emph{Laplacian} of $G$. Recall that ${s}$ is the innate opinion vector of all nodes in $G$.
It is not too hard to see that \cref{eq:updateruleFJ} is equivalent to

\begin{align}\label{updaterulelinalg}
    {z}^{(t)} = (D + I)^{-1}(A{z}^{(t-1)} + {s})
    \nonumber
\end{align}

where $ {z}^{(t)} = \begin{bmatrix} z_1^{(t)} & z_2^{(t)} & \cdots & z_n^{(t)} \end{bmatrix}$. 

Let ${z}^{(*)}$ be the final equilibrum opinion vector after $t$ time steps. Then 

\begin{align}\label{zstar}
    {z}^{(*)} = (L + I)^{-1}{s}.
    \nonumber
\end{align}

