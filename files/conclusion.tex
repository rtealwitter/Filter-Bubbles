% chktex-file 46

\section{Conclusion}\label{sec:conclusion}
Filter bubbles are an important ethical issue.
Improving the mathematical theory behind their formation
presents an interesting and applicable area of research.
In this work, we empirically analyzed several
variations on the standard FJ dynamics opinion formation process
including the introduction of two natural
administrator actions.
Our theoretical contribution is a series of extensions
to the convergence bounds given by \cite{chitra20analyzing}.
Notably, we apply the bounds to any mean-centered
innate opinion vector and tightly characterize the
constraints when the Stochastic Block Model is in
the special case of the Erdős–Rényi graph.

\subsection{Future Work}

While answering some questions, our work points to several
avenues for future research:
The power of the theoretical bounds come from the natural
linear-algebraic description of FJ dynamics.
It would be interesting to write a network
administrator action in terms of a matrix multiplication.
Our main obstacle in this pursuit was balancing the
`macro' scale of any linear-algebraic action with
the `micro' scale employed in practice by social media platforms.
Bimodality is a natural and interesting alternative
measure of polarization.
However, the bimodality coefficient involves the third and fourth moments
of expectation.
While polarization can easily be represented as a quadratic measure,
cubic and quartic functions are more difficult to represent
in terms of vector operations.
An avenue for future research is either finding a
different measure or writing bimodality in a neat expression.

In terms of theory, we envision a way to give an explicit
expression for the bounds on the polarization convergence
in terms of the size of the network $n$.
We did not have sufficient time to write this up before
the deadline but hope to extend it later.

\subsection{Reflection and Responsibilities}
We enjoyed working on a problem at the intersection
of theoretical computer science and ethical issues.
It was satisfying to apply mathematical tools
to a problem facing the modern world.
Our greatest challenge was thinking of linear algebraic formulations of network administrator actions and bimodality; we ended up focusing on experimental results for these ideas, and have left them as interesting avenues for future work. 
However, we feel that we extended existing theoretical results
and introduced interesting variations to this problem, which we studied
empirically. 
It is our hope that we can publish this work or a future
iteration of it.
Finally, we thank Professor Chris Musco and Raphael Meyer for discussing this project with us and providing input on generating theoretical results. 


Responsibilities for the project and report:
Indu focused on related work, background, and processing
the data we used while
Teal coded the experiments and came up with the
theoretical extensions.
We collaborated on the abstract ideas of each section
and worked together to move through challenging parts
in the theoretical analysis.
