\section*{Proposal}

\subsection*{Motivation and Previous Work}

On the surface, the prominence of social media seems like it could make 
the world more connected, and expose users to a dizzying variety of diverse
ideas. However, recent studies have suggested social media encourages the
opposite, separating individuals into groups unable to find agreement, thereby
polarizing society. A popular explanation for this phenomenon has been coined 
the filter bubble-- the idea that the content displayed on 
users’ feeds is social
media networks is simply an echo chamber that prevents people from accessing a
diverse array of viewpoints. Since user feed content is constrained by metrics
that aim to increase user engagement and ad revenue (i.e. friends’ and 
followers’
views, internet search history, user location, etc.), social media companies
explicitly incentivize users to pay preferential attention to like-minded
content. In this manner, users end up living in a ‘filter bubble’ of their own
ideas. Filter bubbles have been blamed for the spread of misinformation in
Brexit, the 2016 U.S. presidential election, and increased 
distrust in democracy.

Chitra and Musco’s paper aimed to provide a rigorous mathematical theory
solidifying filter bubbles’ emergence, studying user opinion dynamics within
social networks \cite{chitra_analyzing_2020}.
Each network is modeled as a weighted graph per a \emph{stochastic
block model}. Each node corresponds to an individual, and edges between nodes
correspond to social relationships. The “stochastic” aspect is as follows: the 
probability of two nodes being connected is higher 
when the nodes are in the same
community, and lower in different communities. Relationships between users with
increased interactions (i.e. their respective stories pop up in their news feeds
more often) have higher edge weights; relationships where users barely interact
are assigned low edge weights. Chitra and Musco 
adapted the \emph{Friedkin-Johnsen model}
to address opinion dynamics,  augmenting each 
node with an opinion, a real number
in the interval [-1,1]. As time passes, each node’s opinion value is updated
based on the average opinion of their neighbors and connections in the social
network. Polarization is represented by the variance of opinions within the
network, where opinions are represented as a $n$-dimensional vector.

To formalize social media companies’ roles in creating filter bubbles,
Chitra and Musco studied the influence of an important outside actor on 
the Friedkin-Johnsen model: the \emph{network administrator}. The network
administrator’s job is to minimize disagreement among users by 
modifying the edge weights of the graph such that users interact
with more content from users with similar opinions. The network
administrator’s modifications are subject to certain constraints;
for instance, they cannot change the degree of any vertex, 
and they can only modify edge weights by a small amount. 
Chitra and Musco ran experiments on the social networks Twitter
and Reddit to simulate the effect of a network administrator
on polarization.

Their experimental results confirm filter bubble theory. 
That is, a social network modeled by the stochastic block 
model is, with high probability, already in a state of \emph{fragile consensus}. 
Though a network may exhibit low polarization, a minor change in 
edge weights can cause a shift to high polarization. 
For example, when the network administrator changed only 
$40\%$ of the total edge weight in the graph, polarization 
increased by a $40$-fold factor. Chitra and Musco also demonstrated 
a simple fix to this phenomenon. Recall that the network 
administrator’s object was to minimize disagreement:
$\min_G D_z$.
They proposed a solution to add a regularization term to the objective
(ridge regression): 
$\min_G D_z + \lambda w^Tw$
Their solution constrained the increase in polarization by a network
administrator to $4\%$. 

\subsection*{Research Questions}

We propose the following further research directions to
extend Chitra and Musco’s work. 

\begin{enumerate}
    \item \emph{Formalizing the network administrator’s role}: Chitra and Musco informally assert that the network 
    administrator’s role is to minimize disagreement. Can we prove, formally, that this is the case-- i.e. 
    solve the optimization problem? Could we prove that a “simpler” or “natural” action by the network administrator-- i.e. maintaining the overall weight of the graph, but increasing the weights between highly agreeable neighbors and decreasing the weights between disagreeing neighbors-- still converges 
    to a more highly polarized vector of opinions? Can we prove this in a stochastic block model? What 
    about other models? 
    \item \emph{Dealing with outside attacks}:
    Could we model some sort of 
    attack by an outside actor on increasing polarization? 
    An example is Cambridge Analytica’s role in the 2016 
    presidential election to sway undecided voters toward 
    electing Trump, popularized by the Netflix documentary 
    “The Great Hack.” Here’s a simple idea: at each time step, 
    introduce a new ‘fake’ node with an opinion. How many fake nodes,
    and how strong of opinions, would we have to introduce to 
    converge to a final set of opinions with higher polarization,
    and how many time steps would that take?
    \item \emph{Investigating Bimodality}:
    Some suggest that polarization, 
    defined by Chitra and Musco as the variance of the n opinions 
    of the network’s users, might not actually be what’s increasing
    as much as Chitra and Musco experimentally demonstrate. 
    What’s actually going on may be more accurately explained by increasing \emph{bimodality}, where opinions converge in clusters around two centers, as opposed to opinions being further away from where we started.  In other words, we can modify Chitra and Musco’s question to: “How sensitive are social networks to bimodality?” 
    We propose formally defining bimodality and seeing 
    if we can obtain a similar result to the “fragile consensus”
    linear algebraic results by Chitra and Musco. 
\end{enumerate}
